\documentclass[10pt]{article}
\usepackage{fullpage,enumitem,amsmath,amssymb,graphicx,array}

\begin{document}

\begin{center}
{\Large Project Progress Report}

\begin{tabular}{rl}
SUNet ID: & paulmtz \\
Names: & Paul Martinez, Calvin Studebaker \\
\end{tabular}
\end{center}

\section*{Introduction}

We are interested in the process of automatically generating music playlists given a set of songs. More specifically, we would like to train an algorithm that partitions a set of songs according to some criteria that it has been trained on. After some discussion of various approaches to the problem, we will consider the specific case where we are simply trying to partition a set of songs into two distinct sets. In this case we may phrase the problem as a sort of classification algorithm, but we will explore various other potential approaches along the way.

\section*{Model}

To begin we would like to formally define the problem and the objective function we are trying to maximize. The task we would like to accomplish is to be given a set of songs $S$ and generate $K$ playlists $P_1, P_2, \ldots, P_K$ from the set $S$. These sets $P_1, \ldots, P_K$ will form a partition of the set $S$. The goal is that 


\end{document}
